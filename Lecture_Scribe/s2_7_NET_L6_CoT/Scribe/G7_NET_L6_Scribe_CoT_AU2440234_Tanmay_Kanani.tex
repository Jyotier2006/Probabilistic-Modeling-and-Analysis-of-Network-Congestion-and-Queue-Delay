\documentclass[12pt]{article}

\usepackage{amsmath}
\usepackage{amssymb}
\usepackage{geometry}
\geometry{margin=1in}

\title{CSE400 – Fundamentals of Probability in Computing\\
Lecture 6: Discrete Random Variables, Expectation and Problem Solving}
\author{Instructor: Dhaval Patel, PhD}
\date{January 22, 2025}

\begin{document}

\maketitle

\section*{L6\_S2\_A}

\section{Random Variables}

\subsection{Definition and Concept}

Let $\Omega$ denote the sample space.

$\omega \in \Omega$ denote a sample point.

A random variable $X$ is defined as a function:

\[
X : \Omega \rightarrow \mathbb{R}
\]

That assigns a real number to each sample point $\omega \in \Omega$:

\[
X(\omega) \in \mathbb{R}
\]

Thus, a random variable converts outcomes of an experiment into real numerical values.

\subsection{Discrete Random Variables}

The lecture restricts attention initially to discrete random variables, which satisfy:

They take values from a finite or countably infinite set.

The actual set of values:

\[
\{X(\omega) : \omega \in \Omega\}
\]

forms a discrete subset of the real numbers.

This means the values can be listed explicitly.

\subsection{Visualization of Random Variable Distribution}

The distribution of a discrete random variable can be visualized using a bar diagram:

The x-axis represents possible values of the random variable.

The height of each bar represents:

\[
P(X=a)
\]

Each probability corresponds to the probability of the associated event in the sample space.

\section{Types of Random Variables}

\subsection{Discrete Random Variable}

A random variable is discrete if:

\begin{itemize}
\item It has countable support.
\item It is described using a Probability Mass Function (PMF).
\item Probabilities are assigned to individual values.
\item Each possible value has strictly positive probability.
\end{itemize}

\subsection{Continuous Random Variable}

A random variable is continuous if:

\begin{itemize}
\item It has uncountable support.
\item It is described using a Probability Density Function (PDF).
\item Probabilities are assigned over intervals.
\item Each individual value has probability zero.
\end{itemize}

\section{Example: Tossing Three Fair Coins}

\subsection{Experiment Description}

An experiment consists of tossing 3 fair coins.

Define:

\[
Y = \text{number of heads observed}
\]

Possible values of $Y$:

\[
Y \in \{0,1,2,3\}
\]

\subsection{Computing Probabilities}

The sample space contains 8 equally likely outcomes:

\[
\{TTT, TTH, THT, THH, HTT, HTH, HHT, HHH\}
\]

Each has probability:

\[
\frac{1}{8}
\]

Case 1: No heads

\[
P(Y=0) = P(TTT) = \frac{1}{8}
\]

Case 2: One head

Possible outcomes:

\[
\{TTH, THT, HTT\}
\]

Thus:

\[
P(Y=1) = \frac{3}{8}
\]

Case 3: Two heads

Possible outcomes:

\[
\{THH, HTH, HHT\}
\]

Thus:

\[
P(Y=2) = \frac{3}{8}
\]

Case 4: Three heads

Possible outcome:

\[
HHH
\]

Thus:

\[
P(Y=3) = \frac{1}{8}
\]

\subsection{Validity Condition of PMF}

Since $Y$ must take exactly one of these values:

\[
P(Y=0) + P(Y=1) + P(Y=2) + P(Y=3) = 1
\]

Thus,

\[
\frac{1}{8} + \frac{3}{8} + \frac{3}{8} + \frac{1}{8} = 1
\]

This confirms the distribution is valid.

\section{Probability Mass Function (PMF)}

\subsection{Definition}

A random variable that can take at most countable values is called a discrete random variable.

Let the range of values be:

\[
R_X = \{x_1, x_2, x_3, \dots\}
\]

Define the function:

\[
P_X(x_k) = P(X = x_k)
\]

This function is called the Probability Mass Function (PMF) of $X$.

\subsection{Properties of PMF}

The PMF must satisfy:

Property 1: Non-negativity

\[
P_X(x_k) \ge 0
\]

Property 2: Total probability equals 1

Since the random variable must take one of its possible values:

\[
\sum_{k=1}^{\infty} P_X(x_k) = 1
\]

This follows because the possible values are mutually exclusive and collectively exhaustive events.

\section{Example: Finding Unknown Constant in PMF}

\subsection{Given}

The PMF is:

\[
p(i) = c \frac{\lambda^i}{i!}, \quad i = 0,1,2,\dots
\]

where $\lambda > 0$.

We must find:

\[
P(X=0)
\]

\[
P(X>2)
\]

\subsection{Step 1: Use Total Probability Condition}

Since PMF must sum to 1:

\[
\sum_{i=0}^{\infty} p(i) = 1
\]

Substitute PMF:

\[
\sum_{i=0}^{\infty} c \frac{\lambda^i}{i!} = 1
\]

Factor out constant:

\[
c \sum_{i=0}^{\infty} \frac{\lambda^i}{i!} = 1
\]

Using known identity:

\[
\sum_{i=0}^{\infty} \frac{\lambda^i}{i!} = e^\lambda
\]

Thus:

\[
c e^\lambda = 1
\]

Solve for $c$:

\[
c = e^{-\lambda}
\]

\subsection{Step 2: Find $P(X=0)$}

Substitute into PMF:

\[
P(X=0) = c \frac{\lambda^0}{0!}
\]

\[
= e^{-\lambda}
\]

\subsection{Step 3: Find $P(X>2)$}

Use complement rule:

\[
P(X>2) = 1 - P(X \le 2)
\]

\[
= 1 - [P(X=0) + P(X=1) + P(X=2)]
\]

Substitute:

\[
= 1 - \left(e^{-\lambda} + \lambda e^{-\lambda} + \frac{\lambda^2 e^{-\lambda}}{2}\right)
\]

Thus,

\[
P(X>2) = 1 - e^{-\lambda} \left(1 + \lambda + \frac{\lambda^2}{2}\right)
\]

\section{Bayes’ Theorem}

\subsection{Conditional Probability Relationship}

Using definition of conditional probability:

\[
Pr(A|B) = \frac{Pr(B|A) Pr(A)}{Pr(B)}
\]

This gives Bayes’ Formula:

\[
Pr(B_i | A) =
\frac{Pr(A|B_i) Pr(B_i)}
{\sum_{i=1}^{n} Pr(A|B_i) Pr(B_i)}
\]

\subsection{Terminology}

Prior probability

\[
Pr(B_i)
\]

Probability before observing new information.

Posterior probability

\[
Pr(B_i | A)
\]

Probability after observing event $A$.

\section{Bayes’ Theorem Example: Auditorium}

\subsection{Problem Description}

An auditorium has 30 rows.

Number of seats in rows:

Row 1 $\rightarrow$ 11 seats

Row 2 $\rightarrow$ 12 seats

Row 3 $\rightarrow$ 13 seats

$\dots$

Row 30 $\rightarrow$ 40 seats

Selection procedure:

Step 1: Select a row randomly (equal probability)

\[
P(R_k) = \frac{1}{30}
\]

Step 2: Select a seat randomly within that row.

\subsection{Find $P(S_{15} | R_{20})$}

Row 20 has:

\[
10 + 20 = 30 \text{ seats}
\]

Thus,

\[
P(S_{15} | R_{20}) = \frac{1}{30}
\]

\subsection{Find $P(R_{20} | S_{15})$}

Using Bayes’ formula:

\[
P(R_{20} | S_{15})
=
\frac{
P(S_{15}|R_{20}) P(R_{20})
}
{
\sum_{k=5}^{30}
P(S_{15}|R_k) P(R_k)
}
\]

Since seat 15 exists only in rows 5 through 30.

Denominator:

\[
\sum_{k=5}^{30}
\frac{1}{k+10}
\cdot
\frac{1}{30}
\]

Numerical evaluation from lecture:

\[
P(R_{20} | S_{15}) \approx 0.0325
\]

\section{Summary of Key Results}

Random Variable

\[
X : \Omega \rightarrow \mathbb{R}
\]

PMF

\[
P_X(x_k) = P(X=x_k)
\]

\[
\sum P_X(x_k) = 1
\]

Exponential Identity

\[
\sum \frac{\lambda^i}{i!} = e^\lambda
\]

Bayes’ Theorem

\[
P(B_i | A) =
\frac{
P(A|B_i) P(B_i)
}
{
\sum P(A|B_i) P(B_i)
}
\]

\end{document}
