\documentclass[12pt]{article}
\usepackage{amsmath, amssymb}
\usepackage{geometry}
\geometry{margin=1in}

\begin{document}

\begin{center}
School of Engineering and Applied Science (SEAS), Ahmedabad University\\
CSE 400: Fundamentals of Probability in Computing\\
Lecture Scribe Notes: Random Variables\\
Lecture 6  – Discrete Random Variables, Expectation, and Problem Solving\\
Instructor: Dhaval Patel, PhD\\
Date: January 22, 2025
\end{center}

\section*{1. Random Variables (Recap and Concepts)}

\subsection*{1.1 Definition}
A random variable (RV) on a sample space is a function that assigns a real number to each sample point.  
\[
X : \Omega \rightarrow \mathbb{R}, \quad X(\omega), \ \omega \in \Omega
\]

Discrete Random Variables: We restrict attention to RVs that are discrete. This means they take values in a range that is finite or countably infinite. Even though $X$ maps to $\mathbb{R}$, the set of values $\{X(\omega) : \omega \in \Omega\}$ is a discrete subset of $\mathbb{R}$.

\subsection*{1.2 Probability Mass Function (PMF)}
For a discrete random variable $X$ with a range (possible values) $R_X = \{x_1, x_2, x_3, \ldots\}$, the Probability Mass Function (PMF) is defined as:
\[
P_X(x_k) = P(X = x_k), \quad \text{for } k = 1,2,3,\ldots
\]

Legitimacy Property: Since $X$ must take on one of the values $x_k$, the sum of probabilities must equal 1:
\[
\sum_{k=1}^{\infty} P_X(x_k) = 1
\]

Visualization: The PMF can be visualized as a bar diagram where the x-axis represents the values the variable can take, and the height of the bar is the probability.

\section*{2. Worked Example: Solving for a Constant in a PMF}

The PMF of a random variable is given by
\[
p(i) = c \frac{\lambda^i}{i!}, \quad i = 0,1,2,\ldots
\]
where $\lambda$ is a positive value.

\subsection*{Solution}
For a legitimate PMF,
\[
c \sum_{i=0}^{\infty} \frac{\lambda^i}{i!} = 1
\]
Using the Taylor series expansion
\[
e^x = \sum_{i=0}^{\infty} \frac{x^i}{i!}
\]
we obtain
\[
c e^{\lambda} = 1 \Rightarrow c = e^{-\lambda}
\]

\[
P\{X=0\} = e^{-\lambda}
\]

\[
P\{X>2\} = 1 - P\{X \le 2\}
\]
\[
P\{X>2\} = 1 - P\{X=0\} - P\{X=1\} - P\{X=2\}
\]

\section*{3. Bayes' Theorem and Independent Events}

\subsection*{3.1 Bayes' Theorem}
\[
P(B_i|A) = \frac{P(A|B_i)P(B_i)}{\sum_{i=1}^n P(A|B_i)P(B_i)}
\]

A priori probability: presupposed models.  
Posteriori probability: calculated after observing events.

\subsection*{3.2 Communication System Example}

\[
P(0Rx) = P(0Rx|0Tx)P(0Tx) + P(0Rx|1Tx)P(1Tx)
\]
\[
P(0Rx) = (0.95)(0.5) + (0.10)(0.5) = 0.525
\]

\subsection*{3.3 Independent Events}

Two events $A$ and $B$ are independent if
\[
P(A|B) = P(A)
\]
\[
P(A,B) = P(A)P(B)
\]

\section*{4. Types of Discrete Random Variables}

\subsection*{4.1 Bernoulli Random Variable}

\[
X =
\begin{cases}
1 & \text{Success} \\
0 & \text{Failure}
\end{cases}
\]

\[
P_X(1) = p, \quad P_X(0) = 1-p
\]

\subsection*{4.2 Binomial Random Variable}

\[
X \sim B(n,p)
\]
\[
p(i) = \binom{n}{i} p^i (1-p)^{n-i}, \quad i=0,1,\ldots,n
\]

\subsection*{4.3 Geometric Random Variable}

\[
P(X=n) = (1-p)^{n-1}p
\]

\[
\sum_{n=1}^{\infty} p(1-p)^{n-1} = 1
\]

\subsection*{4.4 Poisson Random Variable}

\[
p(i) = P(X=i) = e^{-\lambda}\frac{\lambda^i}{i!}, \quad i=0,1,2,\ldots
\]

\[
\sum_{i=0}^{\infty} e^{-\lambda}\frac{\lambda^i}{i!} = 1
\]

Poisson random variables can approximate Binomial random variables when $n$ is large and $p$ is small.

\end{document}
