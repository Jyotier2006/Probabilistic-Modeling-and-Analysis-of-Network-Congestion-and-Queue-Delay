\documentclass[12pt]{article}

\usepackage[a4paper,margin=1in]{geometry}
\usepackage{amsmath,amssymb}
\usepackage{setspace}
\usepackage{enumitem}

\setstretch{1.2}

\begin{document}

\begin{center}
\textbf{CSE400 – Fundamentals of Probability in Computing}\\
\textbf{Lecture 6: Discrete Random Variables, Expectation and Problem Solving}\\
Instructor: Dhaval Patel, PhD\\
Date: January 22, 2025\\
\vspace{0.3cm}
\textit{(This scribe is intended for closed-notes / reading-based exam revision.)}
\end{center}

\vspace{0.5cm}

\section*{L6\_S2\_A}

\section{Random Variables}

\subsection{Motivation and Concept}

Let $\Omega$ denote the sample space.

\textbf{Definition:}  
A random variable $X$ on a sample space $\Omega$ is a function
\[
X : \Omega \rightarrow \mathbb{R}
\]
that assigns to each sample point $\omega \in \Omega$ a real number $X(\omega)$.

\subsection*{Restriction to Discrete Random Variables}

Until further notice, attention is restricted to discrete random variables, i.e., random variables that take values in a set that is finite or countably infinite.

Although $X$ maps to $\mathbb{R}$, the actual set of values
\[
\{X(\omega) : \omega \in \Omega\}
\]
forms a discrete subset of $\mathbb{R}$.

\subsection*{Visualization}

Sample points in $\Omega$ are mapped by $X$ onto discrete points on the real line.

Multiple sample points may map to the same real value.

\subsection{Distribution of a Discrete Random Variable}

The distribution of a discrete random variable can be visualized using a bar diagram.

The x-axis represents the possible values of the random variable.

The height of the bar at value $a$ equals:
\[
\Pr[X = a]
\]

Each probability is computed by finding the probability of the corresponding event in the sample space.

\section{Types of Random Variables}

\subsection{Discrete Random Variables}

\textbf{Characteristics:}
\begin{itemize}
\item Countable support
\item Probability Mass Function (PMF)
\item Probabilities assigned to single values
\item Each possible value has strictly positive probability
\end{itemize}

\subsection{Continuous Random Variables}

\textbf{Characteristics:}
\begin{itemize}
\item Uncountable support
\item Probability Density Function (PDF)
\item Probabilities assigned to intervals
\item Each individual value has zero probability
\end{itemize}

\section{Example 1: Tossing 3 Fair Coins}

\subsection*{Experiment Description}

An experiment consists of tossing 3 fair coins.

Let:
\[
Y = \text{number of heads observed}
\]

\subsection*{Possible Values of $Y$}

\[
Y \in \{0,1,2,3\}
\]

\subsection*{Probability Calculations}

\[
P(Y = 0) = P(t,t,t) = \frac{1}{8}
\]

\[
P(Y = 1) = P(t,t,h) + P(t,h,t) + P(h,t,t) = \frac{3}{8}
\]

\[
P(Y = 2) = P(h,h,t) + P(h,t,h) + P(t,h,h) = \frac{3}{8}
\]

\[
P(Y = 3) = P(h,h,h) = \frac{1}{8}
\]

\subsection*{Resulting Distribution}

\[
P(Y=0)=\frac{1}{8},\quad
P(Y=1)=\frac{3}{8},\quad
P(Y=2)=\frac{3}{8},\quad
P(Y=3)=\frac{1}{8}
\]

\subsection*{Normalization Condition}

Since $Y$ must take exactly one of the values $0,1,2,3$:
\[
1 = P\left( \bigcup_{i=0}^{3} \{Y=i\} \right)
= \sum_{i=0}^{3} P(Y=i)
\]

\section{Probability Mass Function (PMF)}

\subsection{Definition}

A random variable that can take on at most a countable number of possible values is called discrete.

Let $X$ be a discrete random variable with range:
\[
R_X = x_1, x_2, x_3, \ldots
\]
(where the range may be finite or countably infinite).

Define:
\[
P_X(x_k) = P(X = x_k), \quad k=1,2,3,\ldots
\]

This function $P_X(\cdot)$ is called the Probability Mass Function (PMF) of $X$.

\subsection{PMF Normalization Property}

Since $X$ must take one of the values $x_k$:
\[
\sum_{k=1}^{\infty} P_X(x_k) = 1
\]

\section{PMF – Worked Example}

\subsection*{Given}

The PMF of a random variable $X$ is:
\[
p(i) = c \frac{\lambda^i}{i!}, \quad i=0,1,2,\ldots
\]
where $\lambda > 0$.

Find:
\begin{enumerate}
\item $P(X=0)$
\item $P(X>2)$
\end{enumerate}

\subsection*{Step 1: Find Constant $c$}

Using normalization:
\[
\sum_{i=0}^{\infty} p(i) = 1
\]

\[
c \sum_{i=0}^{\infty} \frac{\lambda^i}{i!} = 1
\]

Using:
\[
e^{\lambda} = \sum_{i=0}^{\infty} \frac{\lambda^i}{i!}
\]

\[
ce^{\lambda} = 1 \Rightarrow c = e^{-\lambda}
\]

\subsection*{Step 2: Compute $P(X=0)$}

\[
P(X=0) = e^{-\lambda}\frac{\lambda^0}{0!} = e^{-\lambda}
\]

\subsection*{Step 3: Compute $P(X>2)$}

\[
P(X>2) = 1 - P(X \le 2)
\]

\[
= 1 - [P(X=0) + P(X=1) + P(X=2)]
\]

\[
= 1 - \left( e^{-\lambda} + \lambda e^{-\lambda} + \frac{\lambda^2}{2} e^{-\lambda} \right)
\]

\section{Bayes’ Theorem (Recap)}

\subsection{Conditional Probability Identity}

\[
\Pr(A \mid B) = \Pr(B \mid A)\Pr(A)
\]

\subsection{Bayes Formula (Proposition 3.1)}

\[
\Pr(B_i \mid A) =
\frac{\Pr(A \mid B_i)\Pr(B_i)}
{\sum_{j=1}^{n} \Pr(A \mid B_j)\Pr(B_j)}
\]

\subsection{Terminology}

\[
\Pr(B_i): \text{a priori probability}
\]

\[
\Pr(B_i \mid A): \text{posterior probability}
\]

\section{Bayes’ Theorem – Example: Auditorium with 30 Rows}

\subsection*{Problem Description}

Auditorium has 30 rows.

Row 1 has 11 seats, Row 2 has 12 seats, $\ldots$, Row 30 has 40 seats.

A door prize is awarded by:
\begin{enumerate}
\item Randomly selecting a row (each row equally likely)
\item Randomly selecting a seat within that row (each seat equally likely)
\end{enumerate}

\subsection*{Tasks}

\begin{enumerate}
\item Compute the probability that Seat 15 was selected given Row 20 was selected.
\item Compute the probability that Row 20 was selected given Seat 15 was selected.
\end{enumerate}

\subsection*{Step 1: Probability of Selecting Seat 15 Given Row 20}

Since Row 20 has 30 seats:
\[
P(S_{15} \mid R_{20}) = \frac{1}{30}
\]

\subsection*{Step 2: Compute $P(S_{15})$}

Using total probability:
\[
P(S_{15}) = \sum_{k=15}^{30} P(S_{15} \mid R_k)P(R_k)
\]

Each row is equally likely:
\[
P(R_k) = \frac{1}{30}
\]

\[
P(S_{15}) = \sum_{k=15}^{30} \frac{1}{k+10} \cdot \frac{1}{30}
\approx 0.0342
\]

\subsection*{Step 3: Probability Row 20 Given Seat 15}

Using Bayes’ formula:
\[
P(R_{20} \mid S_{15}) =
\frac{P(S_{15} \mid R_{20})P(R_{20})}{P(S_{15})}
\]

\[
= \frac{\frac{1}{30}\cdot \frac{1}{30}}{0.0342}
\approx 0.0325
\]

\end{document}

