\documentclass[12pt]{article}
\usepackage{amsmath, amssymb}
\usepackage{geometry}
\geometry{margin=1in}

\begin{document}
\textbf{Yashika Ashar - AU2440182 - Lecture 6 Scribe}
\section*{Discrete Random Variables, Expectation and Problem Solving}


\subsection*{Random Variables: Motivation and Concept}

A random variable is a numerical description of the outcome of a random experiment.
Formally, a random variable $X$ defined on a sample space $\Omega$ is a function
\[
X : \Omega \rightarrow \mathbb{R}
\]
which assigns a real number $X(\omega)$ to each outcome $\omega \in \Omega$.

In this lecture, we restrict attention to \emph{discrete random variables}. Although the
function maps into $\mathbb{R}$, the actual set of values
\[
\{X(\omega) : \omega \in \Omega\}
\]
is either finite or countably infinite, and hence forms a discrete subset of $\mathbb{R}$.

\subsection*{Distribution of a Discrete Random Variable}

The distribution of a discrete random variable can be visualized using a bar diagram.
The values that the random variable can take are represented on the horizontal axis.
For a value $a$, the height of the bar represents the probability
\[
\Pr[X = a].
\]
Each probability is computed by identifying the corresponding event in the sample space.

\subsection*{Discrete and Continuous Random Variables}

\subsubsection*{Discrete Random Variables}

Discrete random variables have the following characteristics:
\begin{itemize}
    \item Countable support
    \item A probability mass function (PMF)
    \item Probabilities assigned to individual values
    \item Each possible value has strictly positive probability
\end{itemize}

\subsubsection*{Continuous Random Variables}

Continuous random variables have:
\begin{itemize}
    \item Uncountable support
    \item A probability density function (PDF)
    \item Probabilities assigned to intervals
    \item Zero probability at any single point
\end{itemize}

\subsection*{Example: Tossing Three Fair Coins}

Consider an experiment in which three fair coins are tossed. Let $Y$ denote the number
of heads obtained. Then $Y$ can take the values $0,1,2,3$.

\[
P(Y=0) = P(t,t,t) = \frac{1}{8}
\]

\[
P(Y=1) = P(t,t,h),(t,h,t),(h,t,t) = \frac{3}{8}
\]

\[
P(Y=2) = P(t,h,h),(h,t,h),(h,h,t) = \frac{3}{8}
\]

\[
P(Y=3) = P(h,h,h) = \frac{1}{8}
\]

Since $Y$ must take one of these values, we have
\[
1 = \Pr\left[\bigcup_{i=0}^{3} \{Y=i\}\right]
= \sum_{i=0}^{3} P(Y=i).
\]

\subsection*{Probability Mass Function}

A random variable that can take at most a countable number of values is called a discrete
random variable. Let $X$ be a discrete random variable with range
\[
R_X = \{x_1, x_2, x_3, \ldots\},
\]
where the set is finite or countably infinite.

The function
\[
P_X(x_k) = P(X = x_k), \quad k = 1,2,3,\ldots
\]
is called the probability mass function (PMF) of $X$.

Since $X$ must take one of the values $x_k$, it follows that
\[
\sum_{k=1}^{\infty} P_X(x_k) = 1.
\]

\subsection*{PMF Example}

The probability mass function of a random variable $X$ is given by
\[
p(i) = c \frac{\lambda^i}{i!}, \quad i = 0,1,2,\ldots
\]
where $\lambda > 0$.  

Since
\[
\sum_{i=0}^{\infty} p(i) = 1,
\]
we have
\[
c \sum_{i=0}^{\infty} \frac{\lambda^i}{i!} = 1.
\]
Using
\[
e^{\lambda} = \sum_{i=0}^{\infty} \frac{\lambda^i}{i!},
\]
it follows that
\[
c e^{\lambda} = 1 \quad \Rightarrow \quad c = e^{-\lambda}.
\]

Thus,
\[
P(X=0) = e^{-\lambda}.
\]

Further,
\[
P(X > 2) = 1 - P(X \leq 2)
\]
\[
= 1 - [P(X=0) + P(X=1) + P(X=2)]
\]
\[
= 1 - \left[e^{-\lambda} + \lambda e^{-\lambda} + \frac{\lambda^2}{2} e^{-\lambda}\right].
\]

\subsection*{Bayes' Theorem}

Using the identity
\[
P(A \cap B) = P(B \mid A) P(A),
\]
we obtain
\[
P(B_i \mid A) =
\frac{P(A \mid B_i) P(B_i)}
{\sum_{j=1}^{n} P(A \mid B_j) P(B_j)}.
\]

This formula is known as Bayes' Theorem.

Here, $P(B_i)$ is the \emph{a priori} probability and
$P(B_i \mid A)$ is the \emph{posterior} probability.

\subsection*{Bayes' Theorem Example: Auditorium}

An auditorium has 30 rows of seats. Row 1 has 11 seats, Row 2 has 12 seats, and so on,
up to Row 30 which has 40 seats.

A row is selected uniformly at random, and then a seat is selected uniformly from that row.

The probability that seat 15 is selected given that row 20 is selected is
\[
P(S_{15} \mid R_{20}) = \frac{1}{30}.
\]

The probability that row 20 was selected given that seat 15 was selected is
\[
P(R_{20} \mid S_{15}) =
\frac{P(S_{15} \mid R_{20}) P(R_{20})}{P(S_{15})},
\]
where
\[
P(R_{20}) = \frac{1}{30}
\]
and
\[
P(S_{15}) = \sum_{k=15}^{30} P(S_{15} \mid R_k) P(R_k).
\]

\end{document}
