\documentclass[12pt]{article}

\usepackage{amsmath}
\usepackage{amssymb}
\usepackage{geometry}
\usepackage{setspace}

\geometry{margin=1in}
\setstretch{1.2}

\title{Lecture Scribe: Randomized Min-Cut Algorithm}
\author{Course: CSE400 – Fundamentals of Probability in Computing\\
Instructor: Dhaval Patel, PhD\\
Based strictly on Lecture Slides and Textbook Context}
\date{}

\begin{document}

\maketitle

\section{Min-Cut Problem}

\subsection{Motivation: Why Use Min-Cut?}

The minimum cut (min-cut) algorithm is used to solve problems related to:

\begin{itemize}
\item Network connectivity
\item Network reliability
\item Optimization problems in networks
\end{itemize}

These applications arise because the min-cut identifies the smallest set of edges whose removal disconnects the network.

\subsection*{Applications}

\textbf{(1) Network Design}

The min-cut helps improve communication efficiency and optimize network flow. It is used to find the minimum capacity cut, which identifies bottlenecks in the network.

\textbf{(2) Communication Networks}

The min-cut helps understand network vulnerability to failures. It enables the construction of robust and fault-tolerant networks.

\textbf{(3) VLSI Design}

In Very Large Scale Integration (VLSI), the min-cut algorithm helps partition circuits into smaller components. This reduces interconnectivity complexity.

\section{Definition of Cut and Minimum Cut}

\subsection{Graph Notation}

Let:

\[
G = (V, E)
\]

Where:

\begin{itemize}
\item $V$ = set of vertices
\item $E$ = set of edges
\item $n = |V|$ = number of vertices
\end{itemize}

\subsection{Definition: Cut-Set}

A cut-set is defined as a set of edges whose removal breaks the graph into two or more connected components.

Thus, removing these edges disconnects the graph.

\subsection{Definition: Minimum Cut (Min-Cut)}

Given a graph $G = (V, E)$, the minimum cut problem is defined as finding a cut-set with minimum cardinality.

This means finding the smallest number of edges whose removal disconnects the graph.

\subsection{Sensitivity of Randomized Min-Cut Algorithms}

Min-cut algorithms such as Karger’s algorithm are randomized and sensitive to edge selection.

If the algorithm contracts critical edges early, it may produce a different cut.

Thus, the algorithm’s outcome depends on random choices.

\section{Edge Contraction Operation}

\subsection{Definition: Edge Contraction}

Edge contraction is the primary operation used in randomized min-cut algorithms.

Edge contraction removes an edge and merges its two endpoints into a single vertex.

\subsection{Formal Description}

When contracting an edge $(u, v)$:

\begin{itemize}
\item Step 1: Merge vertices $u$ and $v$ into a single vertex
\item Step 2: Remove the edge $(u, v)$
\item Step 3: Retain all other edges
\item Step 4: Remove self-loops
\item Step 5: Parallel edges may remain
\end{itemize}

\section{Successful and Unsuccessful Min-Cut Runs}

Since randomized algorithms involve randomness, different runs produce different results.

\subsection{Successful Min-Cut Run}

A successful min-cut run occurs when the algorithm correctly identifies the minimum cut of the graph.

Characteristics:

\begin{itemize}
\item Critical edges are not contracted prematurely
\item Final result equals actual minimum cut
\end{itemize}

\subsection{Unsuccessful Min-Cut Run}

An unsuccessful run occurs when the algorithm fails to identify the minimum cut.

Characteristics:

\begin{itemize}
\item Critical edges were contracted early
\item Final cut is not minimum
\end{itemize}

\section{Max-Flow Min-Cut Theorem}

\subsection{Theorem Statement}

The Max-Flow Min-Cut theorem states:

\begin{quote}
``In a flow network, the maximum amount of flow passing from the source to the sink is equal to the total weight of the edges in a minimum cut.''
\end{quote}

\subsection{Definitions Related to Theorem}

Capacity of a cut:

\[
\text{Capacity of cut} = \sum \text{capacities of edges from } X \text{ to } Y
\]

Where $X$ and $Y$ form the partition of vertices.

Minimum cut:

The cut with smallest possible capacity.

Minimum cut capacity:

Capacity of the minimum cut.

Maximum flow:

Largest possible flow from source $S$ to sink $T$.

\subsection{Result of Theorem}

\[
\text{Maximum Flow} = \text{Minimum Cut Capacity}
\]

\section{Deterministic Min-Cut Algorithm (Stoer–Wagner Algorithm)}

\subsection{Theorem Used}

Let $s$ and $t$ be vertices of graph $G$.

Let $G/s,t$ be the graph obtained by merging $s$ and $t$.

Then the minimum cut of $G$ is the smaller of:

\begin{itemize}
\item Minimum $s$–$t$ cut of $G$, or
\item Minimum cut of $G/s,t$
\end{itemize}

\subsection{Explanation}

Case 1: Minimum cut separates $s$ and $t$

Then minimum $s$–$t$ cut is minimum cut.

Case 2: Minimum cut does not separate $s$ and $t$

Then minimum cut exists in contracted graph $G/s,t$.

\subsection{Pseudocode: MinimumCutPhase(G, a)}

Step 1: Initialize

\[
A \leftarrow \{a\}
\]

Step 2: Repeat until $A \neq V$

Step 3: Add most tightly connected vertex to $A$

Step 4: Return cut weight

\subsection{Pseudocode: MinimumCut(G)}

Step 1: While $|V| \ge 1$

Step 2: Choose vertex $a$

Step 3: Run MinimumCutPhase$(G, a)$

Step 4: If cut is smaller than current minimum cut, store cut

Step 5: Merge last two added vertices

Step 6: Repeat

Step 7: Return minimum cut

\section{Randomized Min-Cut Algorithm}

\subsection{Why Randomized Algorithm?}

Randomized algorithms provide:

\begin{itemize}
\item Probabilistic guarantee of success
\item Efficient computation
\item Robust performance
\item Parallelization capability
\end{itemize}

\section{Karger’s Randomized Min-Cut Algorithm}

\subsection{Main Idea}

The algorithm repeatedly performs edge contraction randomly until only two vertices remain.

The edges between these two vertices form the cut.

\subsection{Algorithm Steps}

Step 1: Start with graph $G$

Step 2: Randomly select an edge

Step 3: Contract the edge

Step 4: Repeat until only two vertices remain

Step 5: Remaining edges form the cut

\section{Recursive Randomized Min-Cut Algorithm}

Input:

Graph $G$ with $n$ vertices

Constant $\alpha > 0$

Output:

Cut $C$

If $n \le \alpha^3$, compute minimum cut using brute force.

Else repeat for $i = 1$ to $\alpha$:

Create graph $G'$ by random contraction

Compute $C' =$ RecursiveRandomizedMinCut$(G', \alpha)$

If $|C'| < |C|$, update $C \leftarrow C'$

Return $C$.

\section{Comparison: Deterministic vs Randomized Min-Cut}

\subsection{Deterministic Min-Cut}

Time complexity:

\[
O(VE + V^2 \log V)
\]

Guarantees exact minimum cut.

\subsection{Randomized Min-Cut}

Time complexity:

\[
O(V^2)
\]

Provides minimum cut with high probability.

\section{Probability of Finding Min-Cut}

\[
\text{Probability} \ge \frac{2}{n(n - 1)}
\]

Where $n$ = number of vertices.

\section{Summary of Key Concepts}

\begin{itemize}
\item Min-cut: smallest set of edges disconnecting graph
\item Core operation: edge contraction
\item Deterministic algorithm: Stoer–Wagner algorithm
\item Randomized algorithm: Karger’s algorithm
\item Max-flow min-cut theorem: Maximum Flow = Minimum Cut Capacity
\item Probability of success $\ge \frac{2}{n(n - 1)}$
\end{itemize}

\end{document}
