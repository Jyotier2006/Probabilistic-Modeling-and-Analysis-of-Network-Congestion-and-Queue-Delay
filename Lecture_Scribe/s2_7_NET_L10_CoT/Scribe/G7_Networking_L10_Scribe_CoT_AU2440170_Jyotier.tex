\documentclass[12pt]{article}

\usepackage{amsmath}
\usepackage{amssymb}
\usepackage{geometry}
\usepackage{setspace}

\geometry{margin=1in}
\setstretch{1.2}

\title{Lecture Scribe \\ CSE400 – Fundamentals of Probability in Computing \\ Lecture 10: Randomized Min-Cut Algorithm}
\date{}

\begin{document}

\maketitle

\section{Min-Cut Problem}

\subsection{Why Use Min-Cut?}

The min-cut algorithm is used in applications related to:

\subsubsection{Network Connectivity, Reliability, and Optimization}

The min-cut algorithm helps analyze and optimize networks by identifying the minimum number of edges whose removal disconnects the graph. This helps understand the structure and robustness of networks.

\subsubsection{Network Design}

Min-cut helps improve communication efficiency and optimize network flow by identifying the minimum capacity cut in a network.

\subsubsection{Communication Networks}

Min-cut helps identify vulnerabilities in networks. It is useful for building robust and fault-tolerant communication systems.

\subsubsection{VLSI Design}

In Very Large Scale Integration (VLSI), min-cut is used to partition circuits into smaller components, reducing interconnectivity complexity.

\section{Definition of Min-Cut}

\subsection{Cut-Set}

\textbf{Definition:} A cut-set in a graph is a set of edges whose removal breaks the graph into two or more connected components.

\subsection{Min-Cut Problem}

Given: A graph

\[
G = (V, E)
\]

where

\[
V = \text{set of vertices}
\]

\[
E = \text{set of edges}
\]

\[
n = |V|
\]

Problem Statement:

The minimum cut (min-cut) problem is to find a cut-set with minimum cardinality.

This means finding the smallest number of edges whose removal disconnects the graph.

\subsection{Edge Contraction Operation}

The main operation in min-cut algorithms is edge contraction.

\textbf{Definition:} Edge contraction is an operation that removes an edge and merges its two endpoints into one vertex.

Formally, when contracting an edge $(u,v)$:

Step 1: Merge vertices $u$ and $v$ into a single vertex

Step 2: Remove all edges connecting $u$ and $v$

Step 3: Retain all other edges in the graph

Result:

Parallel edges may be created

Self-loops are removed

\section{Successful and Unsuccessful Min-Cut Runs}

Min-cut algorithms, especially randomized ones, may produce different outcomes in different runs.

\subsection{Successful Min-Cut Run}

\textbf{Definition:} A successful min-cut run refers to the outcome of the algorithm correctly identifying the minimum cut of the graph.

The diagram on page 6 illustrates a sequence of contractions that leads to the correct minimum cut.

\subsection{Unsuccessful Min-Cut Run}

\textbf{Definition:} An unsuccessful min-cut run refers to an iteration where the algorithm fails to identify the true minimum cut.

This occurs when critical edges belonging to the minimum cut are contracted early.

The diagram on page 7 shows a contraction sequence resulting in an incorrect cut.

\section{Max-Flow Min-Cut Theorem}

\subsection{Theorem Statement}

\textbf{Max-Flow Min-Cut Theorem:}

\[
\text{Maximum Flow} = \text{Minimum Cut Capacity}
\]

This means:

The maximum amount of flow from source to sink is equal to the total weight of edges in the minimum cut.

\subsection{Definitions}

\textbf{Capacity of a Cut}

Capacity of a cut is defined as:

\[
\text{Capacity} = \sum \text{capacity of edges from vertices in set } X \text{ to set } Y
\]

\textbf{Minimum Cut}

Minimum cut is defined as:

The cut with smallest possible capacity.

\textbf{Minimum Cut Capacity}

Minimum cut capacity is:

The capacity of the minimum cut.

\textbf{Maximum Flow}

Maximum flow is:

The largest possible flow from source $S$ to sink $T$.

\section{Deterministic Min-Cut Algorithm}

\subsection{Stoer-Wagner Min-Cut Algorithm}

Let:

\[
G \text{ be a graph}
\]

\[
s, t \text{ be two vertices}
\]

\[
G/\{s,t\} \text{ be the graph obtained by merging } s \text{ and } t
\]

\textbf{Theorem}

A minimum cut of $G$ is equal to the smaller of:

Minimum $s$-$t$ cut of $G$

Minimum cut of $G/\{s,t\}$

\textbf{Reason}

Two possible cases exist:

\textbf{Case 1:} There exists a minimum cut separating $s$ and $t$

Then,

Minimum $s$-$t$ cut is the minimum cut of $G$

\textbf{Case 2:} No minimum cut separates $s$ and $t$

Then,

Minimum cut of $G/\{s,t\}$ is the minimum cut of $G$

\subsection{Pseudocode: MinimumCutPhase}

\textbf{Algorithm: MinimumCutPhase(G, a)}

Step 1: Initialize

\[
A \leftarrow \{a\}
\]

Step 2: While $A \neq V$

Add the most tightly connected vertex to set $A$

Step 3: Return cut weight as cut-of-the-phase

\subsection{Pseudocode: MinimumCut(G)}

Step 1: While $|V| \geq 1$

Step 2: Choose vertex $a \in V$

Step 3: Execute MinimumCutPhase(G, a)

Step 4: If cut-of-phase is smaller than current minimum cut

Update minimum cut

Step 5: Shrink graph by merging last two vertices added

Step 6: Repeat until finished

Step 7: Return minimum cut

\subsection{Time Complexity}

Stoer-Wagner Algorithm complexity:

\[
O(V \cdot E + V^2 \log V)
\]

where

\[
V = \text{number of vertices}
\]

\[
E = \text{number of edges}
\]

\section{Randomized Min-Cut Algorithm}

\subsection{Why Randomized Algorithms?}

Randomized algorithms provide probabilistic guarantees and may improve efficiency.

Advantages:

Efficiency

Parallelization capability

Approximation guarantees

Avoidance of worst-case instances

Heuristic nature

Robustness

\section{Karger’s Randomized Min-Cut Algorithm}

This algorithm repeatedly performs random edge contractions.

The output cut depends on random choices.

Sometimes correct minimum cut is obtained, sometimes not.

\section{Recursive Randomized Min-Cut Algorithm}

\textbf{Algorithm: Recursive-Randomized-Min-Cut(G, $\alpha$)}

\textbf{Input}

Undirected multigraph $G$ with $n$ vertices

Integer constant $\alpha > 0$

\textbf{Output}

Cut $C$

\textbf{Step-by-Step Algorithm}

Step 1:

If

\[
n \leq \alpha^3
\]

then

\[
C \leftarrow \text{min-cut found using brute force}
\]

Step 2:

Else repeat for

\[
i = 1 \text{ to } \alpha
\]

Step 2.1:

Construct $G'$ by applying

\[
n - \left\lceil \frac{n}{\alpha} \right\rceil
\]

random contractions

Step 2.2:

Compute

\[
C' \leftarrow \text{RecursiveRandomizedMinCut}(G', \alpha)
\]

Step 2.3:

If

\[
i = 1 \text{ OR } |C'| < |C|
\]

then

\[
C \leftarrow C'
\]

Step 3:

Return $C$

\section{Comparison: Deterministic vs Randomized Min-Cut}

\subsection{Deterministic Min-Cut}

Properties:

Guarantees exact minimum cut

Higher time complexity for large graphs

Time complexity:

\[
O(V \cdot E + V^2 \log V)
\]

\subsection{Randomized Min-Cut}

Properties:

Provides approximate minimum cut with high probability

Faster algorithm

Time complexity:

\[
O(V^2)
\]

\section{Theorem: Probability of Finding Minimum Cut}

The randomized algorithm outputs a minimum cut with probability at least:

\[
\frac{2}{n(n-1)}
\]

where

\[
n = \text{number of vertices}
\]

\section{Summary of Lecture}

Key concepts covered:

Definition of cut-set and minimum cut

Edge contraction operation

Successful and unsuccessful min-cut runs

Max-Flow Min-Cut Theorem

Stoer-Wagner deterministic algorithm

Randomized min-cut algorithm

Recursive randomized algorithm pseudocode

Comparison between deterministic and randomized methods

Probability theorem for minimum cut detection

\end{document}
