\documentclass[11pt]{article}

\usepackage[a4paper,margin=1in]{geometry}
\usepackage{amsmath, amssymb}
\usepackage{hyperref}
\usepackage{enumitem}

\title{CSE400 -- Fundamentals of Probability in Computing\\
Lecture 10: Randomized Min-Cut Algorithm}
\author{Instructor: Dhaval Patel, PhD}
\date{February 5, 2026}

\begin{document}

\maketitle

\section{Min-Cut Problem}

\subsection{Why Use Min-Cut?}

The min-cut algorithm is used in various applications to solve problems related to:

\begin{itemize}
\item Network connectivity
\item Reliability
\item Optimization
\end{itemize}

Specific applications discussed:

\begin{enumerate}
\item \textbf{Network Design} \\
Min-cut helps in improving the efficiency of communication and optimizing network flow.
The algorithm is used in network design to find the \textbf{minimum capacity cut}.

\item \textbf{Communication Networks} \\
Min-cut helps in understanding the vulnerability of networks to failures.
It is useful for building \textbf{robust and fault-tolerant communication networks}.

\item \textbf{VLSI Design} \\
In Very Large Scale Integration (VLSI) design, the min-cut algorithm is useful for:
\begin{itemize}
\item Partitioning circuits into smaller components
\item Reducing interconnectivity complexity
\end{itemize}
\end{enumerate}

Reference mentioned in lecture: \textit{Section 1.5, Application: A Randomized Min-Cut Algorithm, Probability and Computing, 2nd Edition}

\subsection{What Is Min-Cut?}

\subsubsection*{Definition: Cut-Set}

A \textbf{cut-set} in a graph is a set of edges whose removal breaks the graph into two or more connected components.

\subsubsection*{Definition: Min-Cut}

Given a graph
\[
G = (V, E)
\]
with $n$ vertices, the \textbf{minimum cut (min-cut) problem} is to find a \textbf{minimum cardinality cut-set} in $G$.

\subsection{Edge Contraction (Core Operation)}

\subsubsection*{Definition: Edge Contraction}

Edge contraction is an operation that:

\begin{itemize}
\item Removes an edge from a graph
\item Simultaneously merges the two vertices connected by that edge
\end{itemize}

\textbf{Step-by-Step Description of Contracting an Edge $(u,v)$:}

\begin{enumerate}
\item Merge vertices $u$ and $v$ into a single vertex
\item Eliminate all edges connecting $u$ and $v$
\item Retain all other edges in the graph
\item The resulting graph:
\begin{itemize}
\item May contain parallel edges
\item Contains no self-loops
\end{itemize}
\end{enumerate}

This operation is repeatedly applied in min-cut algorithms.

\section{Successful and Unsuccessful Min-Cut Runs}

\subsection{Successful Min-Cut Run}

\textbf{Definition}

A successful min-cut run refers to the success in the outcome of an algorithm designed to find the minimum cut in a graph.

The lecture illustrates this using a figure where the sequence of edge contractions preserves the true minimum cut until the final step.

\subsection{Unsuccessful Min-Cut Run}

\textbf{Definition}

An unsuccessful min-cut run refers to an iteration of a min-cut algorithm where the algorithm fails to correctly identify the minimum cut of a given graph.

This failure occurs when critical edges are contracted too early, preventing recovery of the true minimum cut.

\section{Max-Flow Min-Cut Theorem}

\subsection{Statement of the Theorem}

The \textbf{Max-Flow Min-Cut Theorem} states:

\begin{quote}
In a flow network, the maximum amount of flow passing from the source to the sink is equal to the total weight of the edges in a minimum cut.
\end{quote}

\subsection{Definitions Used in the Theorem}

\begin{itemize}
\item \textbf{Capacity of a cut} \\
The sum of the capacities of edges in the cut that are oriented from a vertex
\[
\in X \quad \text{to a vertex} \quad \in Y
\]

\item \textbf{Minimum cut} \\
The cut in the network that has the smallest possible capacity

\item \textbf{Minimum cut capacity} \\
The capacity value of the minimum cut

\item \textbf{Maximum flow} \\
The largest possible flow from source $S$ to sink $T$
\end{itemize}

The theorem establishes equality between:
\[
\text{Maximum Flow} = \text{Minimum Cut Capacity}
\]

\section{Deterministic Min-Cut Algorithm}

\subsection{Stoer--Wagner Min-Cut Algorithm}

\textbf{Theorem Statement}

Let $s$ and $t$ be two vertices of a graph $G$.
Let $G/\{s,t\}$ be the graph obtained by merging $s$ and $t$.

Then:

A minimum cut of $G$ can be obtained by taking the smaller of:

\begin{itemize}
\item A minimum $(s,t)$ cut of $G$
\item A minimum cut of $G/\{s,t\}$
\end{itemize}

\textbf{Reasoning (As Given in Lecture)}

\begin{itemize}
\item Case 1: There exists a minimum cut of $G$ that separates $s$ and $t$ \\
$\rightarrow$ A minimum $(s,t)$ cut of $G$ is a minimum cut of $G$

\item Case 2: No minimum cut separates $s$ and $t$ \\
$\rightarrow$ A minimum cut of $G/\{s,t\}$ gives the minimum cut of $G$
\end{itemize}

\subsection{Pseudocode: Deterministic Min-Cut}

\textbf{Algorithm 1: MinimumCutPhase($G,a$)}

\begin{enumerate}
\item Initialize:
\[
A \leftarrow \{a\}
\]
\item While $A \neq V$:
\begin{itemize}
\item Add to $A$ the most tightly connected vertex
\end{itemize}
\item Return the cut weight as the cut of the phase
\end{enumerate}

\textbf{Algorithm 2: MinimumCut($G$)}

\begin{enumerate}
\item While $|V| \ge 1$:
\begin{itemize}
\item Choose any $a \in V$
\item Call MinimumCutPhase($G,a$)
\item If the cut-of-the-phase is lighter than the current minimum cut:
\begin{itemize}
\item Store it as the current minimum cut
\end{itemize}
\item Shrink $G$ by merging the two vertices added last
\end{itemize}
\item Return the minimum cut
\end{enumerate}

\section{Randomized Min-Cut Algorithm}

\subsection{Why Randomized Algorithms?}

Randomized algorithms provide:

\begin{itemize}
\item A probabilistic guarantee of success
\item A more accurate estimate of the minimum cut with fewer iterations
\item Efficiency
\item Parallelization
\item Approximation guarantees
\item Avoidance of worst-case instances
\item Heuristic nature
\item Robustness
\end{itemize}

\subsection{Karger’s Randomized Algorithm}

Karger’s algorithm repeatedly performs random edge contractions.

An example run shown in the lecture demonstrates:

\begin{itemize}
\item Random edge choices
\item Possibility of producing a non-minimum cut
\item Sensitivity to which edges are contracted
\end{itemize}

\subsection{Pseudocode: Randomized Min-Cut}

\textbf{Algorithm 3: RECURSIVE-RANDOMIZED-MIN-CUT($G,\alpha$)}

\textbf{Input:}

\begin{itemize}
\item An undirected multigraph $G$ with $n$ vertices
\item An integer constant $\alpha > 0$
\end{itemize}

\textbf{Output:}

\begin{itemize}
\item A cut $C$ of $G$
\end{itemize}

\textbf{Steps:}

\begin{enumerate}
\item If $n \le 3$:
\[
C \leftarrow \text{a min-cut of } G \text{ found using brute force search}
\]

\item Else:
\begin{enumerate}
\item For $i=1$ to $\alpha$:
\begin{enumerate}
\item Construct $G'$ by applying
\[
n - \left\lceil \frac{n}{\sqrt{\alpha}} \right\rceil
\]
random contraction steps on $G$
\item Compute
\[
C' \leftarrow \text{RECURSIVE-RANDOMIZED-MIN-CUT}(G',\alpha)
\]
\item If $i=1$ or $|C'| < |C|$ then
\[
C \leftarrow C'
\]
\end{enumerate}
\end{enumerate}

\item Return $C$
\end{enumerate}

\section{Comparison: Deterministic vs Randomized Min-Cut}

\subsection*{General Observation}

The choice of approach depends on the specific problem.

\subsection*{Deterministic Min-Cut}

\begin{itemize}
\item Always guarantees an exact minimum cut
\item May have higher time complexity for large graphs
\item Stoer--Wagner algorithm time complexity:
\[
O(V \cdot E + V^2 \log V)
\]
\end{itemize}

\subsection*{Randomized Min-Cut}

\begin{itemize}
\item Produces an approximate minimum cut with high probability
\item Karger’s algorithm time complexity:
\[
O(V^2)
\]
\end{itemize}

\section{Theorem for Min-Cut Set}

The randomized algorithm outputs a minimum cut set with probability at least:
\[
\frac{2}{n(n-1)}
\]

\section{Python Simulation (Class Activity)}

Students were instructed to:

\begin{itemize}
\item Open the Campuswire post for Lecture 10
\item Download the provided \texttt{.ipynb} file
\end{itemize}

This section was intended for practical demonstration, not theoretical derivation.

\bigskip
\centerline{\textbf{End of Lecture 10}}
\centerline{\textbf{Thank You}}

\end{document}
