\documentclass[12pt]{article}
\usepackage{amsmath,amssymb}
\usepackage{geometry}
\geometry{margin=1in}

\begin{document}

\title{Lecture Scribe Notes: Randomized Min-Cut Algorithm}
\author{Course: CSE400 -- Fundamentals of Probability in Computing \\ Lecturer: Dhaval Patel, PhD}
\date{February 5, 2026}
\maketitle

\section{Introduction to the Min-Cut Problem}

\subsection{Definitions and Terminology}

\textbf{Cut-set:} A set of edges in a graph whose removal breaks the graph into two or more connected components. +2

\medskip

\textbf{Min-Cut Problem:} Given a graph $G=(V,E)$ with $n$ vertices, the problem is to find a cut-set with minimum cardinality. +1

\medskip

\textbf{Edge Contraction:} The fundamental operation used in min-cut algorithms. It involves:
\begin{itemize}
\item Removing an edge $(u,v)$ from the graph. +3
\item Merging the two vertices $u$ and $v$ into a single vertex. +1
\item Eliminating all edges connecting $u$ and $v$ (no self-loops).
\item Retaining all other edges, which may result in parallel edges.
\end{itemize}

\subsection{Applications}

Min-cut algorithms are utilized for network connectivity, reliability, and optimization: +2

\begin{itemize}
\item \textbf{Network Design:} To find the minimum capacity cut and optimize network flow and communication efficiency. +1
\item \textbf{Communication Networks:} To understand network vulnerability to failures and build fault-tolerant systems. +1
\item \textbf{VLSI Design:} To partition circuits into smaller components, reducing interconnectivity complexity.
\end{itemize}

\section{Max-Flow Min-Cut Theorem}

The theorem establishes the relationship between flow and cuts in a network.

\medskip

\textbf{Statement:} In a flow network, the maximum amount of flow passing from the source ($S$) to the sink ($T$) is equal to the total weight of the edges in a minimum cut. +2

\medskip

\textbf{Key Components:}
\begin{itemize}
\item \textbf{Capacity of a cut:} The sum of capacities of edges oriented from a vertex in set $X$ to a vertex in set $Y$.
\item \textbf{Minimum cut:} The network cut with the smallest possible capacity.
\item \textbf{Maximum flow:} The largest possible flow from $S$ to $T$.
\end{itemize}

\section{Deterministic Min-Cut: Stoer-Wagner Algorithm}

\subsection{Theorem for Deterministic Approach}

Let $s$ and $t$ be two vertices of graph $G$. Let $G/\{s,t\}$ be the graph obtained by merging them. A minimum cut of $G$ is the smaller of:
\begin{itemize}
\item A minimum $s$--$t$-cut of $G$ (if the min-cut separates $s$ and $t$).
\item A minimum cut of $G/\{s,t\}$ (if the min-cut does not separate $s$ and $t$).
\end{itemize}
+1

\subsection{Stoer-Wagner Pseudocode}

\textbf{Algorithm 1: MinimumCutPhase$(G,a)$}
\begin{itemize}
\item Start with set $A=\{a\}$.
\item While $A \neq V$, add the most tightly connected vertex to $A$.
\item Return the cut weight of the last phase.
\end{itemize}

\medskip

\textbf{Algorithm 2: MinimumCut$(G)$}
\begin{itemize}
\item While $|V| \ge 1$:
\item Choose any vertex $a$ from $V$.
\item Run MinimumCutPhase$(G,a)$.
\item If the cut-of-the-phase is lighter than the current minimum cut, store it as the new minimum cut.
\item Shrink $G$ by merging the two vertices added last in the phase.
\item Return the minimum cut found.
\end{itemize}

\section{Randomized Min-Cut: Karger's Algorithm}

\subsection{Rationale for Randomization}

Randomized algorithms provide a probabilistic guarantee of success and can provide accurate estimates with fewer iterations. Benefits include efficiency, parallelization, approximation guarantees, and avoidance of worst-case instances. +3

\subsection{Karger's Algorithm Mechanism and Example}

The algorithm is sensitive to the initial choice of edges. If critical edges are contracted early, the algorithm may fail to find the true minimum cut. +1

\medskip

\textbf{Example of an Unsuccessful Run:}

\medskip

\textbf{Input Graph:} Vertices $\{0,1,2,3\}$ with edges $a,b,c,d,e$. +2

\medskip

\textbf{Step 1:} Pick edge $b$, remove it, and fuse corners into vertex $\{0,2\}$. +1

\medskip

\textbf{Step 2:} Pick edge $d$, remove it, and fuse corners into vertex $\{1,3\}$. +1

\medskip

\textbf{Outcome:} The output cut is $\{a,c,e\}$, which is not minimal.

\medskip

\textbf{Actual Minimal Cuts:} Either $\{b,e\}$ or $\{a,d\}$.

\subsection{Karger's Recursive Pseudocode}

\textbf{Algorithm 3: RECURSIVE-RANDOMIZED-MIN-CUT$(G,\alpha)$}

\medskip

\textbf{Input:} Undirected multigraph $G$ with $n$ vertices; integer constant $\alpha>0$.

\medskip

\textbf{Base Case:} If $n \le \alpha^3$, find min-cut via brute force.

\medskip

\textbf{Recursive Step:}
\begin{itemize}
\item For $i=1$ to $\alpha$:
\item Create multigraph $G'$ by applying $n-\lceil n/\alpha \rceil$ random contraction steps.
\item Recursively call the algorithm on $G'$ to get cut $C'$.
\item If $i=1$ or $|C'|<|C|$, update $C=C'$.
\end{itemize}

\medskip

\textbf{Output:} Cut $C$.

\subsection{Success Probability Theorem}

The algorithm outputs a min-cut set with a probability of at least:
\[
\frac{2}{n(n-1)}
\]

\section{Comparison: Deterministic vs. Randomized}

\begin{center}
\begin{tabular}{|l|c|c|}
\hline
\textbf{Feature} & \textbf{Deterministic (Stoer-Wagner)} & \textbf{Randomized (Karger's)} \\
\hline
Guarantee & Always guarantees an exact minimum cut. +1 & Provides an approximate min-cut with high probability. \\
\hline
Time Complexity & $O(VE+V^2\log V)$. +1 & $O(V^2)$. \\
\hline
Scalability & May have higher complexity for large graphs. +1 & Generally more efficient for large datasets. \\
\hline
\end{tabular}
\end{center}

\end{document}
