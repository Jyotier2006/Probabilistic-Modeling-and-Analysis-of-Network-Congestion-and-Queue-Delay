\documentclass[11pt]{article}
\usepackage[a4paper,margin=1in]{geometry}
\usepackage{amsmath,amssymb}
\usepackage{enumitem}
\usepackage{setspace}
\setlength{\parskip}{6pt}
\setlength{\parindent}{0pt}

\begin{document}

\begin{center}
{\Large \textbf{CSE400 – Lecture 10 Scribe}}\\
{\large \textbf{Randomized Min-Cut Algorithm}}
\end{center}

\section*{1. Min-Cut Problem — Motivation}

Why use min-cut?

The lecture states:

Min-cut is used in applications related to network connectivity, reliability, and optimization.

We now explain each application step-by-step as given:

\subsection*{(a) Network Design}
\begin{itemize}
\item Goal: improve efficiency of communication
\item Also helps in optimizing network flow
\end{itemize}

Reasoning:

$\rightarrow$ A minimum cut finds the smallest set of edges whose removal disconnects the network

$\rightarrow$ This identifies the minimum capacity cut

$\rightarrow$ Used to design networks with efficient flow

\subsection*{(b) Communication Networks}
\begin{itemize}
\item Used to study vulnerability to failures
\end{itemize}

Reasoning:

$\rightarrow$ If a small cut disconnects the network easily

$\rightarrow$ the network is fragile

Thus:

$\rightarrow$ Min-cut helps build robust and fault-tolerant networks

\subsection*{(c) VLSI Design}
\begin{itemize}
\item Used to partition circuits into smaller components
\end{itemize}

Reasoning:

$\rightarrow$ Min-cut minimizes interconnectivity complexity

$\rightarrow$ reduces wiring and layout complexity

\section*{2. What is a Cut-Set?}

Definition (from slide):

A cut-set in a graph is a set of edges whose removal breaks the graph into two or more connected components.

Logical meaning:
\begin{enumerate}
\item Start with a connected graph
\item Remove some edges
\item If the graph becomes disconnected
\item That edge set is a cut-set
\end{enumerate}

\section*{3. Min-Cut Problem}

Given:

A graph \(G=(V,E)\) with \(n\) vertices

Goal:

Find a minimum cardinality cut-set

Meaning:
\begin{itemize}
\item Among all possible cut-sets
\item choose the one with the smallest number of edges
\end{itemize}

That is the minimum cut.

Important observation from lecture:

Min-cut algorithms like Karger’s are random and sensitive to early edge choices.

Reasoning:
\begin{itemize}
\item If the algorithm contracts a critical edge early
\item the true minimum cut may be destroyed
\item resulting in a larger cut
\end{itemize}

Thus:

$\rightarrow$ randomness can cause success or failure

\section*{4. Edge Contraction — Core Operation}

The main operation used in randomized min-cut is:

Edge Contraction

Definition:

Remove an edge \((u, v)\) and merge vertices \(u\) and \(v\) into one vertex.

Step-by-step:
\begin{enumerate}
\item Pick an edge \((u, v)\)
\item Merge \(u\) and \(v\) into a single vertex
\item Remove all edges between \(u\) and \(v\)
\item Keep all other edges connected to the merged vertex
\item Parallel edges may appear
\item No self-loops remain
\end{enumerate}

Purpose:

Each contraction:

$\rightarrow$ reduces number of vertices

$\rightarrow$ preserves some cuts probabilistically

\section*{5. Successful vs Unsuccessful Min-Cut Runs}

\textbf{Successful Min-Cut Run}

Definition:

A run where the algorithm correctly finds the minimum cut.

Meaning:
\begin{itemize}
\item No critical edges were contracted early
\item The final two super-nodes represent the true min-cut
\end{itemize}

\textbf{Unsuccessful Min-Cut Run}

Definition:

A run where the algorithm fails to identify the true minimum cut.

Reasoning:
\begin{itemize}
\item Important min-cut edges were contracted
\item so the cut disappears
\item final cut is larger than minimum
\end{itemize}

\section*{6. Max-Flow Min-Cut Theorem}

Statement (from slide):

In a flow network, the maximum flow from source to sink equals the total weight of edges in a minimum cut.

Definitions explained:
\begin{itemize}
\item Capacity of a cut = sum of capacities of edges crossing the cut
\item Minimum cut = cut with smallest capacity
\item Maximum flow = largest possible flow from source \(S\) to sink \(T\)
\end{itemize}

Logical relation:

Max possible flow you can push

= smallest bottleneck separating \(S\) and \(T\)

Thus:

Max Flow = Min Cut Capacity

\section*{7. Deterministic Min-Cut Algorithm (Stoer–Wagner)}

Idea:

Let \(s\) and \(t\) be two vertices.

Form graph \(G/\{s,t\}\) by merging \(s\) and \(t\).

Then:

The minimum cut of \(G\) is the smaller of:

a minimum \(s\)-\(t\) cut in \(G\)

a minimum cut in the merged graph

Reasoning:
\begin{itemize}
\item Either the min cut separates \(s\) and \(t\)
\item or it doesn’t
\end{itemize}

If it does → it’s an \(s\)-\(t\) cut

If not → merging doesn’t destroy it

Thus one of them must be the true minimum.

\textbf{Pseudocode Structure (from slide)}

Algorithm 1: MinimumCutPhase(G, a)

\begin{enumerate}
\item Start with \(A = \{a\}\)
\item Repeatedly add the most tightly connected vertex to \(A\)
\item When all vertices added → the cut weight of this phase is returned
\end{enumerate}

Algorithm 2: MinimumCut(G)

\begin{enumerate}
\item Repeatedly run MinimumCutPhase
\item Track smallest cut seen
\item Merge last two vertices of each phase
\item Continue until one vertex remains
\item Return smallest cut found
\end{enumerate}

Time Complexity (from slide):

\(O(V\cdot E + V^2 \log V)\)

\section*{8. Why Randomized Min-Cut?}

Randomized algorithms provide:
\begin{itemize}
\item probabilistic guarantee of success
\item fewer iterations
\item efficiency
\item parallelization
\item robustness
\item avoidance of worst-case instances
\end{itemize}

\section*{9. Karger’s Randomized Min-Cut Algorithm}

Core idea:

Repeatedly apply random edge contraction until only two vertices remain.

Remaining parallel edges = cut size.

Why random?

Each contraction:

$\rightarrow$ preserves the minimum cut with some probability

$\rightarrow$ repeated trials increase chance of success

\section*{10. Recursive Randomized Algorithm (from pseudocode)}

Input:
\begin{itemize}
\item Undirected multigraph \(G\)
\item Integer \(\alpha > 0\)
\end{itemize}

Output:
\begin{itemize}
\item A cut \(C\)
\end{itemize}

Step-by-step:

If \(n \le 3\):

$\rightarrow$ compute min-cut using brute force

Else:

Repeat \(\alpha\) times:

Perform about \(n - n/2\) random contractions

Obtain smaller graph \(G'\)

Recursively compute min-cut \(C'\) of \(G'\)

Keep the smallest cut among all trials

Return the smallest cut found.

\section*{11. Deterministic vs Randomized — Comparison}

Deterministic Min-Cut:

Always exact

Higher time complexity for large graphs

Stoer-Wagner complexity:

\(O(VE + V^2 \log V)\)

Randomized Min-Cut:

Approximate with high probability

Faster for large graphs

Karger’s algorithm complexity:

\(O(V^2)\)

\section*{12. Success Probability Theorem}

The algorithm outputs a minimum cut with probability at least:

\[
\frac{2}{n(n-1)}
\]

Meaning:
\begin{itemize}
\item For one run, probability is small
\item But repeating many times increases success greatly
\end{itemize}

Thus:

$\rightarrow$ randomized approach works by repetition

\section*{Final Logical Flow Summary}

\begin{enumerate}
\item Min-cut identifies weakest edge separation
\item Used in networks, reliability, VLSI
\item Cut-set disconnects graph
\item Min-cut = smallest such set
\item Randomized algorithm contracts edges
\item Success depends on avoiding critical edges
\item Deterministic guarantees correctness but slower
\item Randomized is faster with probability
\item Repetition ensures high success
\end{enumerate}

\end{document}
